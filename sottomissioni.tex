\documentclass{article}
\usepackage{graphicx} % Required for inserting images
\usepackage{xcolor}
\usepackage{hyperref}
%QUESTI SONO I COMANDI PER I COMMENTI
\newcommand{\sla}[1]{{\color{blue}(SL: {#1})}} 
\newcommand{\tla}[1]{{\color{orange}(TL: {#1})}} 

\title{Sottomissioni}
\begin{document}
\maketitle
\begin{center}
    \textbf{LM1 Pagina bianca}
\end{center}
\textbf{Minchia 1}\\
\\
\textbf{ENG: }If it were easy to transpose my thoughts onto this blank page, I would not be compelled to do so.
Were it of any use to see through this pencil, perhaps I would surrender my eyes for a blank page.\\
If this pencil could reveal clearly defined shapes, God could trace a line around a tree with a stick.\\
Were God to trace that line, I would wish to reclaim my eyes, for I could no longer see through the pencil.\\
It is not the fault of the doves that the chickens cannot fly.
Forgive me if what I see appears vague.\\
I beg your pardon, but I cannot contribute to the virtue of your blank pages.\\
All that is born follows this line.\\
If the world was just, it would be illegal to spit our pain onto a blank page, lest someone risk mistaking it for a strange kind of wealth. On this blank page, I am naked, yet I am a dove. I have given my eyes in order to avoid distinguishing amidst this chaos. The pencil chooses for me what to release. It is indeed a princess.
but you have forgotten that your eyes are the blank page, not the pencil. The princess stabs you if you do not live for her. In this tangle of causes and effects, of emotions and convictions, I choose to live and die outside the line and within it, I choose, for I would do so even by not doing it, and therefore, it is of no consequence that you carry with you this blank page.\\
\\
\textbf{Minchia 2}\\
\\
\textbf{ITA: }Avevo paura. Paura di fermarmi. Paura di rimanere, per errore, in un luogo sconosciuto, vuoto di ogni riferimento. Eppure, era proprio quella paura, quel giorno, a muovermi.\\
\\
Quell'uomo sulla quarantina, gravemente stempiato, che mi figurava davanti, non lo conoscevo. Eppure gli parlavo con naturalezza, come si fa con un amico, senza troppi filtri. Lui era libero, lo avevo sentito dai discorsi degli altri. Libero di non fermarsi. Dovevo conoscerlo. Volevo capire come facesse.\\
\\
Allora iniziammo a parlare, aiutati dal fatto che anche lui, come me, era italiano.
Ma ciò che mi iniziò a dire era complesso. Mi parlava di idee. Non entrava mai nei particolari, non mi spiegava mai lo scopo o il perché di tutto ciò. Non volle neppure raccontare come quelle idee gli fossero balenate in testa.\\
\\
Si accorse che non capivo, che non stavo centrando il punto. Le mie domande erano troppo specifiche, troppo precise a cui neanche lui, padre di quelle idee, avrebbe saputo rispondere. Perché quelle erano solo idee, nient'altro.
Allora, intraprese un discorso ancor più astratto. Parlava di creatività e di come gestirla. Parlava di tempo.\\
\\
L'idea era il seme e per fare crescere la pianta ci vuole del tempo, e lui non ne aveva. Me le stava regalando, voleva che me ne prendessi cura io. Voleva aiutarmi nel movimento. Così che dalle sue idee potessi imparare a gestire anche le mie. Ero la sua pagina bianca.\\
\\
\textbf{You}\\
\\
\textbf{ENG: }Sometimes when i lay somewhere doing nothing at all i stare at things surrounding me and i try to
align them with my sight. I close one of my eyes and try to match the inclination of,say, the balcony
railing and the corner of the building behind it, tilting my head and moving it just a little so that
the two things overlap, or until one follows the other in the same exact line. It creates a nice sense
of perfect and i feel kind of in control of what’s happening in the world just by staying still, doing
nothing at all. I also do it sometimes when i am listening to someone talking to me, never mentioned
to them, i wonder if they recognize me doing it, closing one of my eyes and tilting my face just a little
so that things i look at perfectly overlap.\\
When there’s a blank page in front of me on the desk i begin to draw weird expressionist Picasso-like
faces on it one after the other until i sometimes completely cover the whole page with them. I especially
do it when I’m on the phone for a long call or when listening to lectures online. It helps me focus
on the thing I’m actually doing, and the best drawings come when I’m not actually thinking about
making them.\\
Frank Zappa wrote ”The Black Page”, the title suggests an extremely difficult and dense piece of music
that the sheet it is written on is completely full of notation making the page seem almost black. I
guess that the opposite of it would be Composition 1960 n.7 by La Monte Young, a piece of music
with only two notes on it played together with the instruction: ”To be held for a long time”. Can one
do better than that? John Cage has of course done it. But can a page still be blank and produce some
transmissible information? Apart from perfect silence, or the random fluctuations from silence, like in
the 4:33 piece?
When the page is left blank what we are left with is our body. Oral tradition is a big thing in the world,
in particular in music. But tradition is some other way to imprint information into something, which
in this case is human memory itself. What I’m really interested in is the instantaneous production
of information. The main questions: How complex an instantaneously produced information can be
with respect to every written one or imprinted in some other way? Can a tale invented on the spot be
complex as every other written book? Could you instantaneously compose a black page?\\
When the page is left blank what is really left is our body, and our imagination, and some other
mysterious thing that is so important that i cannot truly speak about it, let alone write.\\
\\
\textbf{ITA: }A volte quando me ne sto seduto da qualche parte a non fare assolutamente niente mi metto a
fissare le cose che mi circondano e cerco di allinearle con lo sguardo. Chiudo un occhio e cerco di
far corrispondere, tipo, la ringhiera del balcone e lo spigolo del palazzo dietro, inclinando la testa e
muovendola giusto un poco finch ́e le due cose non sono perfetamente sovrapposte, o fin quando una
segue l’altra nella stessa identica linea. Mi crea un piacevole senso di perfezione e mi sento come se
fossi in controllo di quello che accade nel mondo solamente stando fermo, non facendo assolutamente
niente. A volte lo faccio quando sto parlando con qualcuno, mai detto in realt`a, mi chiedo se si rendano
conto che lo sto facendo, chiudere un occhio e inclinare la testa giusto un poco di modo che le cose che
guardo si sovrappongano esattamente.\\
Quando c’`e un foglio bianco di fronte a me comincio a disegnarci strane facce Picassiane espressioniste
fin quando a volte non copro tutta la pagina. Lo faccio specialmente quando sono in una lunga
chiamata al telefono o quando seguo delle lezioni online. Mi aiuta a concentrarmi sulla cosa che sto
facendo realmente, e i migliori disegni vengono fuori quando non ci sto veramente pensando.\\
Frank Zappa ha scritto ”The Black Page”, il titolo suggerice una composizione estremante difficile e
densa che il foglio su cui `e scritta `e completamente pieno di notazione e sembra quasi nero. Immagino che l’opposto possa essere ”Composition 1960 n.7” di La Monte Young, un pezzo che consiste in sole
due note suonate contemporaneamente e l’istruzione ”To be held for a long time”. Si pu`o fare di
meglio? John Cage l’ha sicuramente fatto. Ma pu`o una pagina bianca produrre qualche informazione
trasmissibile? A parte il perfetto silenzo, o le fluttuazioni casuali del silenzio, come in ”4:33”?\\
Quando la pagina rimane bianca quello che rimane `e il nostro corpo. La tradizione orale `e una cosa
importante nel mondo, specialmente nella musica. Ma la tradizione `e un qualche tipo di meccanismo per
imprimere informazione su qualcosa, in questo caso la memoria umana stessa. Quello che mi interessa
davvero `e la produzione istantanea dell’informazione. La domanda principale: Quando pu`o essere
complessa un informazione prodotta istantaneamente rispetto a un’informazione scritta o impressa in
qualche altro modo? Pu`o una storia inventata sul momento essere complessa come un qualsiasi altro
racconto scritto? Si pu`o comporre istantaneamente una black page?\\
Quando la pagina rimane bianca quello che rimane `e il nostro corpo, e la nostra immaginazione,
e qualche altra cosa misteriosa che `e cos`ı importante che non posso davvero parlarne, figuriamoci
scriverne.\\
\\
\textbf{Edward Mani di Forbice}\\
\\
\textbf{ITA: }Sopra questa pagina bianca\\
ci vengo.\\
Con questa pagina bianca,\\
ci sparo.\\
Su questa pagina bianca dormo.\\
Con la pagina bianca,\\
mi taglio,\\
e diventa rossa.\\
\\
\textbf{ENG: }Over this white page\\
I cum.\\
With this white page,\\
I shoot with it.\\
On this white page I sleep.\\
With the white page,\\
I cut myself,\\
and it turns red.\\
\\
\textbf{Tommasum}\\
\\
\textbf{ITA: }La pagina bianca mi ha sempre spaventato. Il giorno del tema mi alzavo sapendo che sarebbe stata una mattinata dolorosa. Forse era il dover creare senza una avere guida, senza indicazioni precise e dettagliate (non riuscivo mai a farmi ispirare dalle tracce), forse era la paura di scrivere qualcosa di estremamente banale; e in realtà l’avrei pure scritta qualche banalità, pur di mettere fine a quella tortura della pagina bianca, intonsa, mentre gli altri sembravano aver già tirato fuori mezzo tema. Ma figuriamoci se uno orgoglioso come me avrebbe mai consegnato alla prof un lavoro volutamente mediocre. \\
\\
Poi arrivò quel momento orrendo in cui iniziarono a darci quattro, cinque, sei ore per scrivere i temi, “per prepararvi alla maturità”, dicevano. La prima ora la passavo metà a guardare nel nulla, metà a disperarmi perché passavano i minuti e io non avevo ancora scelto la traccia. Quando finalmente mi decidevo, arrivava il momento peggiore di tutti: l’assenza totale di idee. Anche la traccia migliore faceva schifo; o forse, semplicemente, non mi andava di scrivere. Restava il fatto che la seconda ora era produttiva quanto la prima. Sconfitto, buttavo giù due righe, prontamente cancellate: solo la prima di una lunga serie di cancellature sporche, maldestre. Mi guardavo a sinistra, poi a destra, e tutti erano già sul retro del foglio. Una tragedia.\\
\\
E poi, all’improvviso, eccola: la parvenza di un’idea, un concetto, qualcosa. iniziavo a scrivere. Con calma, ovviamente; una parola su tre veniva rimossa, modificata, corretta. Poi, a metà della pagina bianca, un avvenimento magico, quasi sovrannaturale: il tema prendeva vita. Quello che avevo intenzione di scrivere non aveva alcuna importanza, non ero io a decidere. Le parole uscivano dalla penna una dopo l’altra, più rapide della mia mente, e, sorprendentemente, erano sempre le parole giuste. Mi fermavo solo per rileggere le righe migliori, quelle di cui già andavo fiero un attimo dopo averle scritte. Così, mentre gli altri iniziavano a preparare la bella (croce e delizia della mia prof, non facevo mai in tempo a scriverla), io tiravo fuori idee che non sapevo neanche di avere. Ed ecco che dopo quattro, cinque, sei lunghissime ore, riuscivo finalmente a concludere il mio tema.\\
\\
Il bello è che, alla fine, mi divertivo come un pazzo. Non so come, i miei temi erano spesso pezzi di realismo magico, cosa che ho scoperto anni dopo, visto che al liceo non ne avevo ancora letto di realismo magico. Erano belli quei temi, mi piacevano proprio. Sono ancora la cosa di cui vado più fiero.\\
\\
\textbf{ENG: }The blank page has always scared me. In school, on the day of the essay, I would wake up knowing it was going to be a painful morning. Maybe it was the need to create without guidance, without precise and detailed instructions (I could never get inspired by the prompts), or maybe it was the fear of writing something extremely banal. Honestly, I would have written some banalities just to put an end to that torture of the blank, untouched page, while others seemed to have already written half their essay. But there was no way someone as proud as me would ever hand in a deliberately mediocre piece of work to the teacher.\\ 
\\
Then came that awful moment when they started giving us four, five, six hours to write essays, "to prepare you for the final exams," they said. I would spend the first hour half staring into space, half despairing because the minutes were passing and I still hadn’t chosen a prompt. When I finally decided on one, the worst moment of all arrived: the total absence of ideas. Even the best prompt sucked; or maybe, simply, I didn’t feel like writing. Nevertheless, the second hour was as unproductive as the first. Defeated, I would jot down a couple of lines, only to immediately erase them: the first of a long series of messy, clumsy deletions. I would look to the left, then to the right, and everyone was already on the back of their sheet. A tragedy.\\ 
\\
And then, suddenly, there it was: the glimmer of an idea, a concept, something. I would start writing. Slowly, of course; one word out of three would be removed, modified, corrected. Then, halfway through the blank page, a magical, almost supernatural event would occur: the essay would come to life. What I had intended to write no longer mattered, I wasn’t the one calling the shots. The words flowed from the pen one after the other, faster than my mind, and, surprisingly, they were always the right words. I would only stop to reread the best lines, the ones I was already proud of just moments after writing them. So, while others started preparing their final draft (the bane of my teacher’s existence, as I never had time to write one), I would come up with ideas I didn’t even know I had. And there, after four, five, six excruciatingly long hours, I would finally manage to finish my essay.\\
 \\
The funny thing is, in the end, I had a blast. Somehow, my essays often turned into pieces of magical realism, something I discovered years later, since I hadn’t read any magical realism in high school. Those essays were beautiful, I really liked them. They’re still the thing I’m most proud of.\\
\\
\textbf{Parentesi Quadrata}\\
\\
\textbf{ITA: }È un gioco difficile: ha poche regole.
Un rettangolo, un inizio di sentiero. Fine.\\
Sembra forse che il giocare a regole minime, con scarsità di confini, possa essere più
semplice, più masticabile, e invece. Forse lo è per chi sa come correre in linee storte, per chi
sa come si brancola, per chi si gode lo scavare senza sapere che profondità raggiungere.\\
\\
Io, no.\\
Giocatrice pigra, per favore, indirizzatemi. Date una direzione alle mie parole, ditemi dove
devo finire, che cosa devo dire. Per favore. Mi stanco subito, ho già finito.\\
\\
Quando ho smesso di giocare? Da quanto tempo so sempre dove girare, a che parola
andare a capo, dove mettere un punto?\\
\\
Volevo giocare ma\\
Non so come creare onde con le giunture, non so come camminare come un pesce.\\
Mi diverto poco, quasi mai, un gran peccato, una triste perdita.\\
\\
Sai che forse non mi sta più bene che mi si porti a spasso tutti i giorni alla stessa ora,\\
Basta.\\
Punto i piedi e punto a capo,\\
Fine.\\
\\
\textbf{ENG: }It's a difficult game: it has very little rules.\\
A rectangle, the beginning of a path. That’s it.\\
\\
It seems that playing with minimal rules, with just a few boundaries, might be simpler, easier
to chew, yet instead. Maybe it is easier for those who know how to run in crooked lines, for
those who know how to grope, for those who enjoy digging without knowing how deep to
reach.\\
\\
I don’t.\\
I am a lazy player, so please, direct me. Give a direction to my words, tell me where I must
end, what I should say. Please. I get tired quickly, I'm already done.\\
\\
When did I stop playing? For how long have I automatically known where to turn, after which
word to go to a new line, where to put a full stop?\\
\\
I wanted to play but\\
I don't know how to make waves with my joints, I don't know how to walk like a fish.\\
I have very little fun, almost never, a great pity, a sad loss.\\
\\
It could be that I'm not okay with being taken for a walk every day at the same time anymore,
Enough.\\
I'll point my feet and I’ll go to a new line,\\
The end.\\
\\
\textbf{El Señor}\\
\\
\textbf{ENG:}It was a day like any other at the court of Donnohow, and just like every Sunday, when the sun carved the left atrium of the Royal Palace, the court banquet began.

In the center of the hall, opposite the grand door, stretched the royal family's table, while the wide side aisles hosted two large, crowded tables of ministers and high-ranking officials of the realm. The sun had just touched the designated golden tile, signaling to the guests that the meal could begin.

"How are the trade exchanges with Donnowhen going? Have we managed to get the grain? It's been more than six months since it last grew on the island, we'll end up with none left even at Court!" asked the Prince to the Minister of Commerce. He quickly replied, "According to the seventy-fourth letter from the order, it seems they are asking for further specifications on the colors of the leaves of the wheat spikes we requested. It seems they don't have basil green."
"But how is that possible!?" shouted the Prince. "Everyone knows that is the only truly edible grain! I can't give my people other grains, how is it possible that they don't have it?! Send an immediate letter of complaint and don't reply to them for a whole month, they’ll learn next time!"
"Certainly, Your Highness, it's truly outrageous that they proposed those pea-green ones."

"What have they dared..." The Prince's sentence was interrupted by a sudden roar as the door to the hall flung open. All the guests stared in shock at the relatively tiny court servant, gasping and sweating. "Sire, a Dragonroast is attacking the outskirts of Donnohow, intervention is needed..."
"I don't understand how a mere servant dares to speak to me!" the King commanded. "Decapitate him and serve the soup! Quickly, before it gets cold!" The guards immediately carried out the execution on the spot, and as the warm blood poured onto the ground, the steaming soup was served to all the guests.
From the opposite side of the table, a certain murmur began to rise, probably some guests were concerned about the Dragonroast, but now the meal had started, and with it, discussions on more important matters resumed. The minstrels began to play, and the clinking of utensils was soon drowned out by lively conversations. Suddenly, the door opened again, and a second servant, even sweatier than the first, shouted, "The dragon has reached the walls, Captain, assemble the army!"
A heavy silence fell over the room. The Captain of the Guard stared penetratingly. Everyone in the hall was in utter disbelief. No one could understand how a servant had dared to give an order to the Captain of the Guard. This time, just one look was enough, and the servant was immediately decapitated, just like his predecessor. After the blood was cleaned up, the door was closed, and the meal continued.
The Captain, to himself, thought, "Maybe I should call in the army... But what will the King think of me if he sees that I've listened to a mere servant? I'll lose my prestige and goodbye summer palace! No, better to move on to the second course; after all, they can handle themselves, they're excellent soldiers."
When the dish was served to all the guests, strange flashes began to penetrate the stained-glass windows. Suddenly, the entire central vault was painted in carnival colors: strokes of red, yellow, blue, green, orange, and purple reshaped the space, and all the guests were left open-mouthed at the indescribable spectacle. Some began to wonder where such a wonder could have come from, and whether it was somehow connected to the dragon (assuming there was a dragon), but the Prince, to avoid embarrassment, calmed everyone down by explaining that, based on his astronomical knowledge, it was probably the result of a phenomenon known in literature as Prodigybulshittius, and there was nothing to fear.

So, the guests returned to enjoying their meal, even happier than before! It was, after all, time for dessert! However, the Captain's mind was still troubled by the thought of what to do, so before the dessert, he stepped out to the main windowsill to take a look. The entire city of Donnohow was engulfed in flames, the walls were destroyed, the fields burned, but since all the senior officials were at Court, the people continued with their daily routines: some were hanging sausages by the fire, some were hoeing over the embers, others were baking bread over them; armies of soldiers, standing and waiting for orders, were swept away by the Dragonroast.

"What are you doing there, paralyzed? Fight, defend the Palace!" the Captain shouted. The soldiers, now mostly decimated, had finally received the long-awaited order and began to form a defensive line around the Palace, but there was little left to be done, and the Captain understood that it was already all lost.
He returned sadly to the hall, sat at his place, took the dessert spoon, and slowly plunged it into the pannacotta tart before him. The soft surface of the tart gave way smoothly to the metal, and when it reached its limit, it broke with a clean but soft cut. Meanwhile, the luminous display of the Prodigybulshittius continued, all the guests were joyous, and no one remembered the great insults from a few hours earlier.
Slowly, however, a strange smell began to waft through the hall, as though something were baking in the oven. The King asked the court cook if there was a second dessert, but the cook confirmed that the meal was over. The King, suspicious, approached the Captain of the Guard and asked how the dragon situation had ended. The Captain promptly replied, "My lord, no Dragonroast could ever harm your army!"
The King's eyes trembled for a moment, then settled, and he ordered that the coffee be served. With the hissing of the moka pots, as if by magic, the palace walls began to emit steam and dissolve into the air. Of all the kingdom of Donnohow, only the table remained, and at its center, the King, sipping his coffee, bathed in golden light, wished everyone a pleasant continuation.\\
\\
\textbf{ITA: } Era un giorno come tanti altri alla corte di Nonsocome e, proprio come tutte le domeniche, quando il sole scolpiva l'atrio sinistro del Palazzo reale, aveva inizio la mensa cortigiana.  
Centrale nella sala dirimpetto al grande portone si allungava il tavolo della famiglia reale,  mentre le ampie navate laterali accoglievano due grandi tavolate fitte fitte di ministri e alti funzionari del reame. Il sole aveva appena lambito l'apposita mattonella dorata indicando ai commensali che il pranzo poteva iniziare. 
"Come procedono gli scambi commerciali con Nonsoquando? Siamo riusciti a procurarci il grano? Ormai sono più di sei mesi che non ne cresce più sull'isola, finirà che non ne avremo più neanche a Corte!" chiese il Principe al ministro del commercio. Questi prontamente rispose "stando alla settantaquattresima missiva dell'ordine, pare che chiedano ulteriori specifiche sui colori delle foglie delle spighe da noi richieste, sembra che il verde basilico non l'abbiano" "ma come é possibile!?" sbraitò il Principe "tutti sanno che quello é l'unico grano davvero commestibile, non posso dare al mio popolo altri grani, com'é possibile che non l'abbiano?! Manda subito una missiva di lamentele e non rispondergli per un mese intero, così imparano meglio la prossima volta" "certamente signor Principe, é davvero oltraggioso che ci abbiano proposto quelle verde pisello" "cosa hanno osat.." la frase del Principe venne interrotta da un improvviso boato, il portone della sala si spalancò e tutti i commensali fissarono attoniti il proporzionalmente minuscolo servo di corte che ansimava tutto sudato "Sire, un Dragarrosto sta attaccando le periferie di Nonsocome, serve interven.." "Non capisco come possa un misero servo permettersi di parlarmi! decapitatelo e versate la zuppa! svelti prima che si raffreddi!" ribadì il Re. Le guardie provvedettero all'esecuzione sul posto e così come il sangue caldo si riversava a terra, la zuppa fumante fu servita a tutti i commensali. Dal lato opposto della tavola si iniziava a percepire un certo brusio, qualcuno probabilmente si stava preoccupando del Dragarrosto, ma ecco che iniziava il pranzo e con esso si riprese a discutere di faccende più importanti, i menestrelli iniziarono a suonare e il tintinnio delle posate fu subito coperto da animati discorsi. D'improvviso il portone s'aprì di nuovo e un secondo servitore, più sudato persino del primo, gridò "Il drago ha raggiunto le mura, Capitano raduni l'esercito!". Il silenzio cadde pesante nella sala. Il Capitano delle Guardie lo guardò penetrante. Tutti in sala erano nel più assoluto sbigottimento, nessuno riusciva a capire come un servo avesse avuto l'ardore di dare un ordine al Capitano delle Guardie. Questa volta bastó solo un occhiata e il servo venne immediatamente decapitato proprio nello stesso luogo del suo predecessore. Pulito il sangue, la porta venne richiusa e il pranzo continuò. Il Capitano, tra sé e sé valutò "forse dovrei far intervenire l'esercito... Ma a questo punto che penserà di me il Re se vedrà che avrò dato retta ad un misero servo? Perderò il prestigio e addio reggia estiva! Nono, meglio passare al secondo, dopotutto sanno cavarsela da soli, sono ottimi soldati". Quando il piatto fu servito a tutti i commensali, strani bagliori iniziarono a penetrare le vetrate colorate. D'un tratto, tutta la volta centrale venne dipinta a carnevale: pennellate di rosso, giallo, blu, verde, arancio, viola ridisegnavano gli spazi e tutti i commensali rimasero a bocca aperta guardando quell' indescrivibile spettacolo. Alcuni iniziarono a domandarsi da dove potesse provenire tale meraviglia e se in qualche modo fosse legata al drago (sempre ammesso che un drago ci fosse), ma il Principe per evitare figuracce calmò tutti spiegando che, date le sue conoscenze astronomiche, era probabilmente frutto di un fenomeno noto in letteratura come Prodigisupercazzulis e non c'era nulla di cui aver paura. Così tutti i commensali tornarono a godersi il pranzo ancora più contenti di prima, era il momento del dolce dopotutto! La mente del Capitano tuttvia era ancora scossa all'idea di dover agire, così prima del dolce uscì sul davanzale principale a dare un'occhiata. Tutta la città di Nonsocome era avvolta dalle fiamme, le mura erano distrutte, i campi bruciati, ma poiché tutti i maggiori funzionari erano a Corte, il popolo continuava imperterrito nel quotidiano: chi appendeva salami al fuoco, chi zappava sulle braci, chi cuoceva il pane; eserciti di soldati immobili in attesa di ordini venivano spazzati via dal Dragarrosto. "Cosa fate lì impalati, combattete, difendete il Palazzo!" gli urlò il Capitano, i soldati ormai già decimati avevano finalmente ottenuto il tanto atteso ordine e iniziarono a schierarsi attorno al Palazzo in formazione difensiva, ma ormai poco era rimasto da fare e il Capitano comprese che era già tutto perduto. Rientrò mestamente nella sala, si sedette al suo posto, prese il cucchiaino da dessert e piano piano lo affondò nel tortino di pannacotta di fronte a lui. La soffice superficie del tortino si piegò compatta al contatto col metallo e, quando raggiunse la curvatura limite, si spezzò con un netto quanto morbido taglio. Intanto lo spettacolo luminoso del Prodigisupercazzulis continuava, tutti i commensali erano gioiosi e nessuno ormai si ricordava più degli enormi oltraggi di qualche ora prima. Piano piano però iniziò ad aleggiare uno strano odore per la sala, come se qualcosa stesse cuocendo in forno, così il Re domandò al cuoco di corte se fosse previsto un secondo dessert, ma il cuoco confermò che il pranzo era finito. Il Re, insospettito, si avvicinò al Capitano dell Guardie e gli domandò come fosse finita la faccenda del drago, così il Capitano prontamente rispose "mio signore, nessun Dragarrosto potrebbe mai nulla contro il vostro esercito!" gli occhi del Re vibrarono per un istante, poi si quietarono e ordinò che il caffè venisse servito. Assieme allo sfiatare delle moke, come per magia anche le pareti del palazzo iniziarono a sprizzare vapore e dissolversi nell'aria. Di tutto il reame di Nonsocome solo la tavola rimaneva, e al centro di essa il Re, che sorseggiando il suo caffè e illuminato di bagliori dorati, auguró a tutti quanti un piacevole proseguimento.\\
\\
\textbf{Ludwig Wichsenstein}\\
\\
\text{ITA: }Non vi è una circostanza chiara in cui il primo paolino nacque, perché se l’essere paolino fosse una qualità innata allora ogni paolino sarebbe tale, e se invece non lo fosse servirebbero esperti per giudicare se un paolino sia tale oppure no. Nel caso in cui paolini si nasca, tale appellativo non sarebbe più che una diagnosi. Mi pare più interessante considerare il caso in cui paolini non si nasca ma si diventi. Il testo cita che un paolino possa essere detto tale da qualcun’altro, o da se stesso. Il primo paolino, secondo i teorici prepaolinisti, divenne tale, o perché additato come paolino da un non paolino, o perché eletto paolino da una cerchia di paolini, o perché autoproclamatosi tale. Alle tre possibilità menzionate i teorici postpaolinisti sostituirono la possibilità che più paolini siano nati allo stesso tempo denominandosi reciprocamente tali. Tale teoria si avvale di una dialettica secondo la quale la parola paolino non potrebbe acquisire significato senza un qualche tipo di consenso, così che la prima persona a pronunciarla non avrebbe emesso altro che un suono. 
La difficoltà nel comprendere chi sia un paolino indipendentemente dall’uso della parola paolino ci rivela una dualità, il paolinismo coesiste simbioticamente con l’uso della parola paolino. Non in tutti casi è la cerchia di paolini a fare uso della parola paolino, potrebbe infatti essere una cerchia di non paolini a consentire sul fatto che i paolini siano tali. Dinstinguiamo quindi il caso in cui l’uso della parola paolino è frequente tra i paolini stessi dal caso in cui ad usare tale parola sia soltanto una cerchia di non paolini, parleremo nel primo caso di propaolinismo e nel secondo di antipaolinismo. I teorici postpaolinisti arrivarono a congetturare che non esista paolinismo senza il coito del propaolinismo con la sua opposizione l’antipaolinismo, sebbene quasi tutti i teorici aderiscano all’antipaolinismo. La parola paolino, in quanto il frutto di un consenso, potrebbe aver avuto significati significativamente diversi per i diversi paolini. Per evitare che la parola paolino torni ad essere un suono si ricorre all’uso della paolinica, ovvero figura del discorso che permette alle diverse facce della parola paolino di evocarsi reciprocamente in qualcosa che all’occhio appare davvero simile al significato. Un esempio di paolinica è questo stesso testo, che permette alle diverse istanze della parola di incontrarsi entro una stessa pagina. Un altra strada che la parola paolino può prendere è divenire un criterio di somiglianza al paolino più paolino di tutti, il paolinissimo, solidificando la matrice liquida che permette al significato di significare. Le due propagande, quella propaolinista e quella antipaolinista, hanno dunque entramble la forma di una paolinica. Per quanto riguarda questo scritto, l’unico consenso può essere tra me e te che leggi, e questo consenso, voluto ma un po’ forzato, suggerisce che forse c’è di più in queste parole che una semplice pagina sporca. Forse amico lettore, se possiamo intuire il significato di una parola che non è, potrebbe essere, ma non avrà alcun motivo di essere, è perché in fondo le parole ci danno forma almeno quanto noi diamo forma a loro. Così mi specchio nel mio foglio, mi chiedo se sono io che sto mettendo parole su una pagina bianca o se è quella che sta scrivendo su di me.\\
\\
\textbf{ENG: }There exists no clear circumstance in which the first paolino was born, for if being paolino were an innate quality, then every paolino would be such by nature; and if it were not, experts would be required to judge whether one is truly a paolino or not. Should paolini be born and not made, then such an appellation would be no more than a diagnosis. Yet it seems more instructive to consider the case in which one does not become a paolino by birth, but rather by transformation. The text states that one may be called paolino either by another, or by oneself. According to pre-paolinist theorists, the first paolino became such either by being designated as paolino by a non-paolino, by being elected paolino by a circle of paolini, or by self-proclamation. To these three possibilities, post-paolinist theorists added the notion that multiple paolini might have arisen simultaneously, designating one another as such. This theory relies on a dialectic tension wherein the word paolino could not acquire meaning without some form of consensus, such that the first to utter it would have produced nothing more than a sound. The difficulty in discerning who is a paolino independently of the use of the word paolino reveals a duality: paolinism coexists symbiotically with the very utterance of the word paolino. Yet it is not always a circle of paolini that makes use of this term; indeed, it may well be a circle of non-paolini who agree upon the fact that the paolini are such. We thus distinguish the case in which the word paolino is frequently employed among paolini themselves from that in which it is used solely by non-paolini; the former shall be termed propaolinism, the latter antipaolinism. Post-paolinist theorists went so far as to conjecture that paolinism cannot exist without the union of propaolinism and its opposition, antipaolinism, though nearly all theorists adhere to antipaolinism. The word paolino, being the fruit of consensus, may have borne significantly different meanings to different paolini. In order to prevent the word paolino from reverting into a mere sound, one resorts to the use of paolinica, a discoursive figure that allows the various facets of the term to evoke one another, forming something that to the eye appears strikingly similar to meaning. An example of paolinica is this very text, which allows the different instances of the word to meet upon the same page. Another path the word paolino may take is to become a criterion of resemblance to the most paolino of all paolini, the paolinissimo, thus solidifying the fluid matrix that allows meaning to signify. The two propagandas, that of propaolinism and that of antipaolinism, therefore both assume the form of a paolinica. As for this writing, the only consensus that may exist is between me and you, the reader, and this consensus, desired yet somewhat compelled, suggests that there may be more in these words than a simple stained page. Perhaps, dear reader, if we can infer the meaning of a word that is not, that could be, but has no reason to be, it is because, in the end, words shape us at least as much as we shape them. Thus, I gaze upon myself in my page, and I wonder: is it I who place words upon a blank sheet, or is it this sheet that is writing upon me?\\
\\
\textbf{Pirka Kalei}
\\
\textbf{ITA:}Alasc whit hiraxaev ierampas torearar eritosur pafandow ai guerra errrante ie saon. Jonhatan urbè sat tim Adonai paaolinu acania gross Enigma yot orolollollo ut ”dammecars” ocroasi naxissis ’takìm katìm Narrsimha. Orvadoss whila hliilàtcar Omega Walter tombstone oracot reazzzionarz e Acrostic dionysugg yot ennagon titties .Arrax aard clister eku iprimi numberiz ai cicow lin oracolonicus uren dnipèrbor. Arissi mnim ovi nostidre so to ereticz roz irindro n’toz t’noz Hari eeee Solomon hogi Acrosticn digimob orbitur witcha sgidibu. Mnumos ereezy Arisossi Nyarlathotepep ie naonè ghemon hallàe iranian dezertos e Satanong In’zothacrosticuss nadal tarapiotapioca homo et feegan omorai L D S oborust fenrir YoggSardon obou ure rezag eioo youBelzebùbn et satyruseeè. Alapor salus saluz. Tzetze horror eat domus Abaeddon roar kakkaculodor ”lil Ivar” nappa erocita sos oaoa nanneemorettic ta ha eu pu aa pa Elohiom readitin oppnosite not liarv youtru stme ut gags, gags enit shitraumaa tumorted Acrostic proprio Acrostica theg horrisble fucker of reuvolutnanta young orcas uis rthisepvena ca rmessgage eif anoibody twinll iaever obread nit? Pi oam onot psuare. Tn hat iycou sexaist pcomce apuooi gtu essere imsiceuro scihne nesisto ovieo tis dnjust anoistei? roitus kaalma et rommitus Taumaturgus
hoic Evarva ”Noodai” tictac hoe et Warioi Hanien iafflqigem tripel eblond pdespueerados Acrostic gag etipa. Tcolumobus hvgrijwirt einekeon wpaulaner hveinieken igdro trotltola eit poly amorouos gags eterdnally ill. Satan nokkie oracple woiaor her ino taclt erezy rotom tarapio hurts Acrostic ntani tofsxaecdsdcs horror evacui. Wow ormai rinchiudo la demenza .\\
\\
\textbf{ENG:}Alasc whit hiraxaev ierampas torearar eritosur pafandow ai guerra errrante ie saon. Jonhatan urbè sat tim Adonai paaolinu acania gross Enigma yot orolollollo ut ”dammecars” ocroasi naxissis ’takìm katìm Narrsimha. Orvadoss whila hliilàtcar Omega Walter tombstone oracot reazzzionarz e Acrostic dionysugg yot ennagon titties .Arrax aard clister eku iprimi numberiz ai cicow lin oracolonicus uren dnipèrbor. Arissi mnim ovi nostidre so to ereticz roz irindro n’toz t’noz Hari eeee Solomon hogi Acrosticn digimob orbitur witcha sgidibu. Mnumos ereezy Arisossi Nyarlathotepep ie naonè ghemon hallàe iranian dezertos e Satanong In’zothacrosticuss nadal tarapiotapioca homo et feegan omorai L D S oborust fenrir YoggSardon obou ure rezag eioo youBelzebùbn et satyruseeè. Alapor salus saluz. Tzetze horror eat domus Abaeddon roar kakkaculodor ”lil Ivar” nappa erocita sos oaoa nanneemorettic ta ha eu pu aa pa Elohiom readitin oppnosite not liarv youtru stme ut gags, gags enit shitraumaa tumorted Acrostic proprio Acrostica theg horrisble fucker of reuvolutnanta young orcas uis rthisepvena ca rmessgage eif anoibody twinll iaever obread nit? Pi oam onot psuare. Tn hat iycou sexaist pcomce apuooi gtu essere imsiceuro scihne nesisto ovieo tis dnjust anoistei? roitus kaalma et rommitus Taumaturgus
hoic Evarva ”Noodai” tictac hoe et Warioi Hanien iafflqigem tripel eblond pdespueerados Acrostic gag etipa. Tcolumobus hvgrijwirt einekeon wpaulaner hveinieken igdro trotltola eit poly amorouos gags eterdnally ill. Satan nokkie oracple woiaor her ino taclt erezy rotom tarapio hurts Acrostic ntani tofsxaecdsdcs horror evacui. Wow ormai rinchiudo la demenza .\\
\\
\textbf{Il Barone Tartaglia}
\\
\textbf{ITA: }Nel nido delle gazze ladre \\
vago come un fungo sopra ai ceppi,\\
curvando verso luoghi di silenzio \\
abitati da gruppo troppo stretti. \\
Rigonfio il lago, soffia il vento del Nord\
fuoriescono acque verdi che ribollono \\
di gabbiani e relitti. \\
Si scuotono i panni delle gonne \\
cariche di simboli e pezze colorate\\
intorno alle gambe nude.\\
Dondolano come campane \\
che baciano di rintocchi gravidi\\
l'aria e la comparsa delle onde.\\
Il fondo è tornato a galla \\
e la cima è andata sotto. \\
Un bambino riempie le rovine di pigne\\
scagliandole come macigni \\
fra le mani dei giganti.  \\
Proprio dietro sta nascosta \\
una casa nella roccia \\
caffè nero di sorgente \\
servito nel cortile all'ombra \\ 
sotto una vecchia quercia bianca, \\
e maghi attraversatori di montagne \\
in cerca di vagabondi. \\
Mi fermo in collina, sotto la moschea \\
siedo accanto a lapidi di pietra  \\
sottili e ruvide come lingue di gatto, \\ 
l'erba verde che si espande come schiuma \\
al sole, mi poggia sdraiato  \\
sopra un sottile foglio bianco.\\
E il mio corpo si fa retta. \\
\\
\textbf{ENG: } In the nest of thieving magpies\\
I wander like a mushroom over the stumps,\\
bending towards places of silence\\
inhabited by a group too tight.\\
The lake swells, the north wind blows\\
green waters spill out, boiling\\
with seagulls and wreckage.\\
The skirts shake and flutter,\\
laden with symbols and colorful rags\\
around naked legs.\\
They sway like bells\\
kissing with heavy peals\\
the air and the rising waves.\\
The bottom has surfaced\\
and the top has gone beneath.\\
A child fills the ruins with pinecones,\\
hurling them like boulders\\
into the hands of giants.\\
Right behind, hidden away,\\
a house in the rock\\
black spring coffee\\
served in the courtyard's shadow\\
beneath an old white oak,\\
and wizards crossing mountains\\
in search of wanderers.\\
I stop on the hill, beneath the mosque\\
sitting beside stone tombstones\\
thin and rough like cat tongues,\\
the green grass spreading like foam \\
in the sun, lying down\\
on a thin white sheet.\\
And my body becomes straight. \\
\\
\textbf{PB\&Jerry}
\\
\textbf{ENG: }20-12-24, Mitiga complex, Tripoli. The blood began to dry up on the young man’s feet as Osama
‘Almasri’ Njeem rinsed the plastic pipe before putting it back in the cabinet. He grinned slightly as he
thought that the falqa rarely failed and today was no exception; less than 30 minutes of whipping to
force the poor devil yell in agony, call his family using an institutional landline phone and state the
amount so as to make the screams in the background stop. Eventually, every mother or sibling always
conceded to wire whatever sum would put an end to the suffering. While removing his gloves Osama
considered spending his freshly earned 5000 dinars on a pony for his little daughter; she had been
bugging him for three weeks but the thought of her smiling at him in gratitude abruptly shifted to
himself sipping a beer in a seat of the executive section at some big football stadium. He remembered
that he’d be traveling to Europe for work with Adel next month and they had often talked about going
to a big European football match. He was so sold on the idea that he instantly forgot about his
daughter and to untie his latest financial sponsor.\\
18-1-2025, The Hague. His bags were bursting with cash and souvenirs when he tried to book a car
from Munich to Rome. The move raised suspicion in the German police which, after little research,
alerted the ICC. As soon as the news reached the upper floors of the Palais de la Paix, Iulia, Maria
and Reine started digging into Mr Njeem’s pile of evidence collected since the previous October. By
22 o’clock they managed to put together an arrest warrant that would - at least - do some justice to Mr
Njeem’s survivors. Had Maria not refused to sign it and had she not attached her own detailed
counter-arguments, they would have been done a good eighty minutes earlier; Reine would never
forgive her for it.\\
19-1-2025, Turin. In the first hours of the day the chief of the Libyan judicial police was still half-awake
in his hotel room when he picked up his phone and he woke up when an unfamiliar voice told him that
he faced an international arrest warrant. The voice also told him to pack his stuff and to be arrested
gentlemanly, that things will be taken care of. Sitting up on the bed in his Vlahovic t-shirt, he stared
blankly, the phone still pressed to his ear.\\
19/20/21-1-205, “Lorusso e Cotugno” prison, Turin. The last time Njeem played with the thought of
ending up behind bars was way back in 2011; given the high military rank he held in the RADA
militias, getting caught by the Gaddafi forces would have most likely meant a swift execution.
European bombs must have been on sale then, he thought gratefully. The RADA turned into the
DACOT forces as soon as the GNA came to power; from then on Osama could kill, torture and rape
without fearing legal consequences–except perhaps a promotion. Until now, he meditated from his cell
bench. Despite feeling reassured by the heads-up call, he was pleasantly surprised at the treatment
he had received in jail: not even a cavity search was a luxury he granted very few of his prisoners,
certainly not women or homosexuals.\\
21-1-25, flying over the Mediterranean. Osama and Adel listened reluctantly to the instructions of
the Italian intelligence officers. The other two Libyan nationals sat quietly on the back seats of the
fighter jet. Osama forgot the reason why he had even brought them with him, though this didn’t matter
now. He’d rather dwell again on the idea that his job had such an impact that he was being shipped
back pronto on an Italian military jet. He tuned back in the conversation and focussed on the angry
looking Italian officer but was only able to hear Almashri we make a mistake but we count on you, you
do a gud job and nobody break your balls. Only dont make them come here, do what you have to do.
You have.. come caz’se dice carta bianca, you have paper white. He didn’t like how confused that little
intercept of a conversation made him so he dozed off until the landing in Tripoli.


\end{document}